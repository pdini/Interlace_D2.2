\chapter{Analysis of Possible Blockchain Technologies for INTERLACE}
\label{ch:dlt}

\vspace{-1cm}
\begin{center}
Paolo Dini and Giuseppe Littera
\end{center}

\section{Basic Concepts}
\subsection{Deterministic Execution}
As discussed in \cite{AndroulakiEtAl2018}, distributed systems have relied on the state machine replication (SMR) model \cite{Schneider1990} for a few decades. The SMR approach is motivated by the need for redundancy in the provision of services to a given client as a strategy to offset possible faults. Each service request from a client is executed in an identical and deterministic way by a set of servers, which means that the same operations are executed in the same order by each server. By definition, the correct response is whatever the majority of the servers calculates. Therefore, SMR is resistant also to Byzantine faults, which are malicious and not merely technical, as long as no more than $t$ servers are affected for a total set comprising $2t + 1$ servers. Most blockchains have taken SMR as a starting point, where the set of operations in this case is one block of transactions. All global consensus algorithms are centred around reaching agreement on the ordering of the transactions in a given block -- hence the name `order-execute' for this type of distributed architecture.

For example, in Bitcoin order-execute involves deterministic sequential execution by each peer, after the first peer has ordered a block, solved the PoW puzzle, and broadcast the block by gossip. The validation of the new block is achieved once every peer has completed and validated the execution. In a public blockchain such as Ethereum a denial-of-service (DoS) attack could be mounted by embedding an infinite loop in the smart contract of one of the transactions. Since it is not possible to determine in general whether or not an algorithm completes (halting problem), such a loop could go undetected, leading to a block that cannot be validated and stopping the ability of the blockchain to support future transactions. Ethereum solves this problem by using ``gas'' which, once converted in the cryptocurrency of that blockchain (ether), results in a charge for the execution of transactions.

The low efficiency of sequential execution can be improved upon with parallel execution of unrelated transactions. However, detecting the possible interdependencies is not trivial. Stellar and Holochain seem to have been able to do that.






\section{Brief Summary of Some DLT Technologies}

The INTERLACE team has looked at a number of blockchain technologies:
\begin{packed_item1}
\item Corda
\item Holochain
\item Stellar
\item Quorum (permissioned Ethereum)
\item Hyperledger Fabric
\end{packed_item1}
In this chapter we analyse and discuss them briefly in turn, emphasising the algorithmic, architectural,  mathematical, or financial aspects that are most pertinent to INTERLACE. The frameworks that have made it into this short list are all interesting for one reason or another, and at different points each of them was seriously taken into consideration for adoption. Stellar has an interesting mathematical foundation which is therefore studied and discussed in some detail since mathematical understanding can be transferred to other frameworks. The one that comes closest to the requirements of the Sardex mutual credit system is Hyperledger. Therefore, after presenting and discussing the others Hyperledger will be analysed and presented in more detail.


\subsection{Corda}
Corda \cite{Hearn2016} is a permissioned distributed database for banking networks. It does not use a blockchain



\subsection{Holochain}
Agent-centric
\cite{HarrisBrownEtAl2018}








\subsection{Stellar}


\subsection{Quorum}

\footnote{\url{https://github.com/jpmorganchase/quorum-docs/blob/master/Quorum_Architecture_20171016.pdf}}



\subsection{Hyperledger}
The main points of interest of Hyperledger Fabric are \cite{AndroulakiEtAl2018}:

\begin{quote}
\begin{packed_item1}
\item It supports modular consensus protocols, which allows the system to be tailored to particular use cases and trust models.
\item Fabric is the first blockchain system that runs distributed applications written in standard, general-purpose programming languages, without systemic dependency on a native cryptocurrency. This stands in sharp contrast to existing blockchain platforms that require �smart-contracts� to be written in domain-specific languages or rely on a cryptocurrency.
\item Fabric realizes the permissioned model using a portable notion of membership, which may be integrated with industry-standard identity management.
\item Fabric achieves end-to-end throughput of more than 3500 transactions per second in certain popular deployment configurations, with sub-second latency, scaling well to over 100 peers.
\end{packed_item1}
\end{quote}



Hyperledger Indy-Plenum Byzantine Fault Tolerance (PBFT).\footnote{\url{https://github.com/hyperledger/indy-plenum/wiki}} PBFT is based on RBFT: Redundant Byzantine Fault Tolerance
\cite{Aublinetal2013}.






%&\hspace{0.5cm}ASM/ASIM framework \\





\begin{sidewaystable}
\small
\begin{centering}
{\begin{tabular}{| l | c | c | c | c | c | c | c | c | c | c |}
\hline
				& 				& \textbf{Read}			& \textbf{Write}		& \textbf{} 
				& \textbf{Smart}		& \textbf{Smart}			&\textbf{Consensus}
				& \textbf{Backup} 	& \textbf{}				&\textbf{Monetary} \\
\textbf{DLT}		&\textbf{Focus}  	& \textbf{Access} 		& \textbf{Access}	& \textbf{Data/Agent} 
				& \textbf{Contracts} 	& \textbf{Contract}		&\textbf{Model} 
				& \textbf{System} 	& \textbf{Interfaces}		&\textbf{Model} \\
				& 				& \textbf{} 				& \textbf{} 			& \textbf{} 
				& \textbf{} 			& \textbf{Language(s)}	&\textbf{} 
				& \textbf{} 			& \textbf{}				&\textbf{} \\
\hline
\hline
\textbf{Ethereum}	&General-purpose		&Public		&Permissionless	&Data-centric	&Yes		&Solidity	
				&Global, PoS 			&?			&?				&Assets \\
				&platform 				&			&				&			&		&on EVM	
				& 					&			&				& \\
\hline
\textbf{Holochain}	&Fog 				&Public		&Permissionless	&Agent-centric	&?		&?
				&Local 				&?			&?				&Mutual Credit\\
\hline
\textbf{Stellar}		& 					&Public		&Permissionless	&Data-centric	&?		&
				&FBAS 				&?			&?				&Assets\\
\hline
\textbf{Corda} 		&Business agreements	&Private		&Permissioned		&Data-centric	&?		&Bytecode
				&Local state			&Relational DB	&SQL			&Assets\\
		 		&between financial inst.	&			&				&			&		&subset on JVM
				&(Notaries)			&			&FPML			&\\
\hline
\textbf{Quorum} 	&Public and  			&Private		&Permissioned		&Data-centric	&?		&
				&BFT 				&?			&?				&Assets\\
			 	&banking sectors		&			&				&			&		&
				& 					&			&				&\\
\hline
\textbf{Hyperledger}	&Enterprise, B2B		&Private		&Permissioned		&Data-centric	&Yes		&JS, Go
				&Centralised 			&Key-value	&?				&Adaptable to\\
 				& \& supply chain		&			&				&			&		&
				& 					&store DB		&				&Mutual Credit\\
\hline
\end{tabular}}

\vspace{1cm}

{\begin{tabular}{| l | c | c | c | c | c | c | c | c | c | c |}
\hline
				& \textbf{}  			& \textbf{}  			& \textbf{Transaction} 	&
				& \textbf{Regulatory/} 	& \textbf{Explicit links}	&\textbf{Business}
				& \textbf{Computational} 	& \textbf{Turing}		&\textbf{Contract} \\
\textbf{DLT}		& \textbf{Blockchain} 	& \textbf{Immutable} 		& \textbf{Validation} 		& \textbf{Architecture} 
				& \textbf{Supervisory} 	& \textbf{of SCs to}		&\textbf{Flow} 
				& \textbf{Model} 		& \textbf{Completeness}	&\textbf{Object} \\
				& \textbf{} 				& \textbf{} 				& \textbf{} 				&
				& \textbf{nodes} 		& \textbf{legal prose}		&\textbf{} 
				& \textbf{} 				& \textbf{}				&\textbf{} \\
\hline
\hline
\textbf{Ethereum}	&Yes			&Yes		&				&Order-Execute	&
				&			&		&Virtual Computer	&No				&Stateful \\
\hline
\textbf{Holochain}	&Individual	&?		&Local to the parties	&				&
				&			& 		&				&				& \\
				&chains		&		&				&				&
				&			& 		&				&				& \\
\hline
\textbf{Stellar}		&			&?		&				&				&?
				&			& 		&?				&?				&\\
\hline
\textbf{Corda} 		&No			&Yes		&Local to the parties	&				&Yes
				&			&Yes		&UTXO			&Yes				&Stateless\\
\hline
\textbf{Quorum} 	&Yes			&?		&				&Order-Execute	&?
				&			& 		&?				&?				&\\
\hline
\textbf{Hyperledger}	&Yes			&Yes		&				&Execute-Order	&
				&			& 		&?				&?				& \\
 				&			&		&				&Validate			&
				&			& 		&				&				& \\
\hline
\end{tabular}}
\caption{\bf \small Comparison of main properties of different types of blockchain}
\label{blockchain_type}
\end{centering}
\end{sidewaystable}





