\chapter{Analysis of Possible Blockchain Technologies for INTERLACE}
\label{ch:dlt}

\vspace{-1cm}
\begin{center}
Paolo Dini and Giuseppe Littera
\end{center}

\section{Introduction}
The INTERLACE team has looked at a number of blockchain technologies:
\begin{packed_item1}
\item Corda
\item Holochain
\item Stellar
\item Quorum (permissioned Ethereum)
\item Hyperledger Fabric
\end{packed_item1}
In this chapter we analyse and discuss them briefly in turn, emphasising the algorithmic, architectural,  mathematical, or financial aspects that are most pertinent to INTERLACE. The frameworks that have made it into this short list are all interesting for one reason or another, and at different points each of them was seriously taken into consideration for adoption. Stellar has an interesting mathematical foundation which is therefore studied and discussed in some detail since mathematical understanding can be transferred to other frameworks. The one that comes closest to the requirements of the Sardex mutual credit system is Hyperledger. Therefore, after presenting and discussing the others Hyperledger will be analysed and presented in more detail.


\section{Corda}
Corda \cite{Hearn2016} is a permissioned distributed database for banking networks. It does not use a blockchain



\section{Holochain}
Agent-centric
\cite{HarrisBrownEtAl2018}








\section{Stellar}


\section{Quorum}

\footnote{\url{https://github.com/jpmorganchase/quorum-docs/blob/master/Quorum_Architecture_20171016.pdf}}



\section{Hyperledger}

\cite{AndroulakiEtAl2018}



Hyperledger Indy-Plenum Byzantine Fault Tolerance (PBFT).\footnote{\url{https://github.com/hyperledger/indy-plenum/wiki}} PBFT is based on RBFT: Redundant Byzantine Fault Tolerance
\cite{Aublinetal2013}.










