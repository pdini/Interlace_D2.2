\chapter{Introduction}
\label{ch:Introduction}

\vspace{-1cm}
\begin{center}
Paolo Dini and Giuseppe Littera
\end{center}

The main purpose of this report was originally to provide a refinement to the specification of the transactional platform as given by deliverable D2.1 \cite{INTERLACE_D21}. That refinement work ended up being integrated in D3.1 \cite{INTERLACE_D31}, because the more urgent challenge of the architecture workpackage became to select an appropriate blockchain technology or framework for the platform. Although this process is being reported now, after the end of the project, the choice was made relatively early on, or in any case early enough for the project to be able to achieve a proof-of-concept implementation. The selection of a blockchain technology for the INTERLACE platform was a long and difficult process because of the wide variety of technologies, architectures, and protocols available; the very rapid rate of innovation in this field; and the evolving requirements of a future vision for the Sardex platform.

Clearly, although INTERLACE is only a first attempt at a blockchain-based transactional platform for the Sardex circuit, it should be as compatible as possible with the future vision of the circuit. In other words, the blockchain framework needs not only to meet certain quality expectations of the technology itself and to satisfy certain business requirements (as described in detail in D3.1), but must also be compatible with the long-term governance and company structure needs of the Sardex initiative as it scales beyond Italy and Europe. In particular, it took a long time to understand how the Sardex mutual credit system could scale to global level while remaining faithful to its principles of support for local economies and of reliance on trust between its users. It is important to realise that the support for local economies and the reliance on trust are not ideological choices. Rather, they should be regarded as a mixture of ethical choice \emph{and} the feature of the network upon which its success is based: that is, the core of the value-generation process, which must therefore be protected.

The fundamental challenge is that trust is most easily built within small social groups and is difficult to maintain in large, long-range systems. Therefore, a global monolithic architecture and company structure could never work because it would lose its value-generating core(s). The answer can only be an architecture that is either distributed or at least hierarchical, with significant delegation and distribution of control to local circuits in local socio-cultural and economic contexts. Since the social and the business dimensions are the starting point for the requirements that lead to choice of technology and architecture definition, we adopted a circular and iterative approach that compared what was possible with what was desirable at each step, and slowly evolved the vision as our understanding of the blockchain space and of our own requirements improved. In this short report we cannot go into too much depth, but it is worth addressing what is arguably the central concept for both Sardex and the blockchain: trust.

\section{Sardex, Trust and the Blockchain}
The blockchain is often described as a `trustless' technology, since the distribution of the validation and record-keeping function to many independent nodes, together with cryptographic algorithms, removes the need for a central authority that plays the role of record keeper and transaction orderer. According to the prevailing view in Computer Science, therefore, trust in a central authority and bilateral trust between transacting parties are substituted for reliance on a type of technology and protocol. Clearly, such an approach is particularly useful in situations where there is no trust at all between transacting parties.

By contrast, the Sardex circuit relies on trust at a fundamental level. Between the ``sociological'' trust discussed by Sartori and Dini \cite{SartoriDini2016} and the trust in the technology platform lies an ``economic'' type of trust. In particular, Sardex relies on, and reinforces, ``thick trust''.\footnote{Richard Simmons, economist, private communication, 2018.} Thick trust has been discussed in the literature in the business context (e.g. \cite{VosselmanMeerKooistra2009}), but for our purposes it is sufficient to define it as the combination of ``Know Your Customer'' with ``Know Their Products''. The role of the Sardex company that runs the circuit, combined with the work of the Sardex brokers, greatly reduces the social cost of trust for the SMEs who participate in the network and achieves and combines both these components, because it is able to make trust transitive: the circuit members trust the Sardex company and the electronic platform, and in the majority of cases this trust extends to bilateral trust between the transacting parties, making the Sardex circuit a particularly strong and stable trading community. One of the drawbacks of such an approach is that the communities and companies involved in the network ultimately depend on one actor to facilitate credit and trade among participants, rendering the network as a whole highly efficient yet vulnerable and not as inclusive and socially/financially adoptable as might be preferred.

Thus, Sardex interested in the blockchain for two reasons. First, as the company operations grow beyond Sardinia and Italy to other countries in Europe and beyond they will involve interactions with other circuits whose legal personality, business relationship with Sardex, and proprietary structure may vary along a range of options depending on the context and stakeholders. Therefore, from a functional and organisational point of view it may be more expedient to build in some flexibility at the level of the architecture: for example, each circuit could run a separate node of the blockchain. This could enable inter-circuit trade via agreements recorded on decentralised public ledgers, more transparency of the overall network, and regional and local clusters of SMEs -- which in turn reduces informational asymmetries.

Second, this organisational flexibility requirement, which is essentially functionalist, is reinforced by the social and cultural requirement of respecting local community identity to the extent possible. This is not so much a matter of institutional or governance efficiency as a question of shared values built around reciprocal respect between different communities who identify with different regions or localities. We feel it is easier to meet such expectations with an articulated and decentralised\footnote{In contrast to what we wrote in the INTERLACE proposal, we have adopted the definition of `decentralised' as an architecture where control is distributed, as opposed to `distributed', which we take to mean only a distribution of functional aspects but leaves the control central.} architecture than through a monolithic platform like Facebook or Google.

\section{Overview of Report}
This brief report provides a high-level discussion of the main blockchain technologies we examined during the course of the project, together with a rationale for choosing Hyperledger Fabric as the core component with possible extensions towards Ethereum and Holochain. The discussion in the rest of the report assumes familiarity with the basic concepts and terminology of the main blockchain technologies as can be found, for example, in \cite{TascaEtAl2017}.

Chapter \ref{ch:dlt} discusses a few candidate Distributed Ledger Technologies (DLTs) that we assessed in the process of deciding which satisfied the INTERLACE and Sardex requirements best. The next step in the process was going to be an ASIM specification and CoreASIM modelling of the transactional platform, which would have been reported in a third chapter, but lack of funding complementary to the INTERLACE budget made this plan impossible. We therefore decided that it made more sense to focus the remaining time and resources on implementing a proof-of-concept transactional platform based on the requirements specified in deliverable D2.1 \cite{INTERLACE_D21} and updated in deliverable D3.1 \cite{INTERLACE_D31}. The implementation work is reported in deliverable D3.2.



\section{Table of Acronyms}
Table \ref{acronyms} shows the definition of the acronyms used in this report.


\begin{table}
\begin{centering}
{\begin{tabular}{| r | c | l |}
\hline
AML		&& Anti Money Laundering\\
\hline
ASM		&& Abstract State Machine \\
\hline
ASIM	&& Abstract State Interaction Machine \\
\hline
B2B		&& Business-to-Business\\
\hline
B2C		&& Business-to-Consumer\\
\hline
BFT		&& Byzantine Fault Tolerance\\
\hline
BTC		&& Currency symbol for Bitcoin\\
\hline
Dapp	&& Distributed App\\
\hline
DB		&& Database\\
\hline
DHT		&& Distributed Hash Table\\
\hline
DLT		&& Distributed (or Decentralised) Ledger Technology\\
\hline
DoS		&& Denial of Service\\
\hline
ETH		&& Currency symbol for Ether, the Ethereum token\\
\hline
EVM		&& Ethereum Virtual Machine\\
\hline
FDAS	&& Federated Distributed Agreement System\\
\hline
FPML	&& Financial Products Markup Language\\
\hline
GDPR	&& General Data Protection Regulation\\
\hline
ICO		&& Initial Coin Offering\\
\hline
JS		&& Javascript\\
\hline
JVM		&& Java Virtual Machine\\
\hline
KYC		&& Know Your Customer\\
\hline
LE		&& Leader Election\\
\hline
PBFT	&& Plenum Byzantine Fault Tolerance\\
\hline
PoA		&& Proof of Authority\\
\hline
PoS		&& Proof of Stake \\
\hline
PoW		&& Proof of Work\\
\hline
QI		&& Quorum Intersection\\
\hline
REST 	&& Representational State Transfer\\
\hline
SC		&& Smart Contract\\
\hline
SCP		&& Stellar Consensus Protocol\\
\hline
SMR		&& State Machine Replication\\
\hline
SQL		&& Structured Query Language\\
\hline
SRD		&& Currency symbol for Sardex credits\\
\hline
Tx/s		&& Transactions per second\\
\hline
UML		&& Unified Modelling Language\\
\hline
UTXO	&& Unspent Transaction Output\\
\hline
XLM		&& Currency symbol for Lumen, the Stellar token\\
\hline
\end{tabular}}
\caption{\bf \small Table of acronyms used in the report}
\label{acronyms}
\end{centering}
\end{table}











\newpage











